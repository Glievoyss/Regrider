\documentclass[12pt]{article}
\usepackage[a4paper,margin=2.5cm]{geometry}
\usepackage{amsmath,amssymb}
\usepackage{graphicx}
\usepackage[utf8]{inputenc}
\usepackage[russian]{babel}
\usepackage{hyperref}
\title{Реграйдерная модель света: волновая интерпретация локальной среды распространения}
\author{Serhii Hliievyi \\ Украина \\ \href{mailto:glievoy.s@gmail.com}{glievoy.s@gmail.com}}
\date{}
\begin{document}
\maketitle
\begin{center}\textbf{Подпись:} Автор концепции векторной волновой модели света в реграйдерной среде\end{center}

Автор: Serhii Hliievyi
Страна: Украина
Контакт: glievoy.s@gmail.com
Подпись: Автор концепции векторной волновой модели света в реграйдерной среде


\section*{Введение}
\addcontentsline{toc}{section}{Введение}

В рамках классической и релятивистской физики распространение света традиционно описывается как процесс, не требующий среды. Согласно специальной теории относительности (СТО), скорость света в вакууме является постоянной и не зависит от движения источника или наблюдателя. Однако данная модель исключает понятие среды как физического носителя волны, что ставит её в логический конфликт с понятием волны как таковой.

В настоящей работе предлагается альтернативный, но совместимый с экспериментами подход: рассмотрение света как волнового явления, распространяющегося в локальной подвижной среде — "реграйдере".

Реграйдер — (от англ. "regrider", совмещение "re-" — вновь, и "grid" — структура, сетка) — это локальная, реагирующая на движение система, в которой распространяется свет. В отличие от исторического "эфира", реграйдер трактуется не как универсальная неподвижная субстанция, а как переносимая, локализуемая волновая среда, связанная с конкретной системой (например, космическим кораблём, поездом и т.д...).

В рамках предложенной модели сохраняется фундаментальный принцип постоянства скорости света "относительно своей локальной среды (реграйдера)", однако общее расстояние, проходимое светом за единицу времени, может варьироваться в зависимости от движения самой среды.

\section*{Физическая аналогия}
\addcontentsline{toc}{section}{Физическая аналогия}

Для наглядности и построения логики модели используется аналогия с распространением звука в движущихся средах. В частности, рассматриваются следующие ситуации:

\par
1. Мужчина стоит на платформе поезда с воздушным шариком и поезд начал движени: шарик не следует за ним мгновенно — аналогично свет не следует за движущимся источником напрямую.
\par
2. Звук, испущенный в движущемся поезде, распространяется в воздухе, который относительно поезда — неподвижен. Следовательно, наблюдатель вне поезда может зарегистрировать различную скорость распространения звука (относительно внешней среды).
\par
3. Мембрана, встроенная в стенку поезда, вибрирует при попадании звуковой волны и порождает волну во внешней среде (воздухе). Для внешнего наблюдателя именно мембрана становится источником звука — как граница между средами. Здесь возникает ключевой момент — точка перехода между двумя средами играет роль нового излучателя.
\par
4. Если убрать мембрану, и звук распространяется напрямую из внутренней среды в наружную, наблюдается аналогичный эффект, как при переходе света из одного реграйдера в другой — например, из космического корабля в вакуум.
\par
Эти шаги постепенно подводят нас к идее, что при переходе между средами изменяется не сама волна, а условия её распространения, и это становится определяющим фактором для внешнего наблюдателя.


\section*{Модель реграйдерной среды}
\addcontentsline{toc}{section}{Модель реграйдерной среды}

Свет рассматривается как волна, распространяющаяся в реграйдере, который может иметь движение относительно внешнего наблюдателя. Внутри реграйдера свет движется со скоростью С, но для внешнего наблюдателя, который фиксирует не только сам свет, но и границу среды, путь света может быть длиннее (или короче), в зависимости от вектора движения реграйдера.

\section*{Математическая модель движения света в реграйдере}
\addcontentsline{toc}{section}{Математическая модель движения света в реграйдере}

Пусть свет проходит участок длиной $\ell$ внутри реграйдера, который движется со скоростью Sk относительно некоторого внешнего наблюдателя. Внутри реграйдера свет всегда распространяется со скоростью C, и, следовательно, время t, необходимое для прохождения этого участка, составляет:

\[
t = \ell / C
\]

В течение этого времени внешнему наблюдателю кажется, что фронт волны смещается не только за счёт движения самого света, но и за счёт движения среды. Таким образом, расстояние, пройденное светом за время t относительно наблюдателя:

\[
L_1 = (C + S_k) \cdot (\ell / C)
\]

После выхода из реграйдера, свет продолжает распространяться в новой среде, которую считаем неподвижной, в течение оставшегося времени:

\[
t_2 = 1 - (\ell / C)
\]

И пройденное расстояние за это время будет:

\[
L_2 = C \cdot t_2 = C \cdot (1 - \ell / C)
\]

Суммарное расстояние, пройденное светом за единицу времени относительно внешнего наблюдателя:

\[
S_t_o_t_a_l = L_1 + L_2 = C + (\ell \cdot S_k) / C
\]

Это ключевое выражение модели: свет может пройти больше или меньше, чем C за 1 секунду, при том, что локальная скорость волны остаётся постоянной.
\par
Если ввести направление движения среды через угол $\theta$ между направлением движения реграйдера и вектором распространения света, то получим обобщённую форму:

\[
S_t_o_t_a_l = C + (\ell \cdot \left|S_k\right| \cdot \cos(\theta)) / C
\]

Значения cos($\theta$) = +1 и -1 являются частными случаями: движение реграйдера строго в направлении света и строго в противоположном соответственно.
\par
Это формирует основу векторной модели распространения света в реграйдерной среде.

\section*{Сравнение с релятивистской моделью и экспериментами}
\addcontentsline{toc}{section}{Математическая модель движения света в реграйдере}
\section*{Эффект Доплера в контексте реграйдерной модели.}
\addcontentsline{toc}{section}{Эффект Доплера в контексте реграйдерной модели.}


Реграйдерная модель объясняет эффект Доплера так же просто, как и поведение звуковой волны в воздухе:

- Когда источник света движется навстречу внешнему наблюдателю (например, на неподвижной платформе), наблюдается "синее смещение".
- Когда источник удаляется — наблюдается "красное смещение".

Однако здесь важно уточнение: сам эффект Доплера возникает не в момент испускания света внутри реграйдера, а при переходе волны через границу среды. Именно в этот момент граница между реграйдерами становится новым условным источником для внешнего наблюдателя, и именно здесь фиксируется изменение частоты.

Для наблюдателя внутри движущегося реграйдера (например, внутри корабля), эффект Доплера отсутствует, поскольку источник света и наблюдатель находятся в одной и той же среде. Скорость света внутри реграйдера для него остаётся равной С, без смещения частоты.

Для внешнего наблюдателя, по отношению к которому движется реграйдер, эффект Доплера наблюдается, несмотря на то, что свет по-прежнему распространяется со скоростью С. Смещение частоты объясняется переходом света из движущейся среды (реграйдера) в неподвижную среду наблюдателя.

Аналогичный эффект можно наблюдать со звуком, покидающим движущийся поезд. Независимо от того, происходит ли выход звука через мембрану или напрямую — наблюдатель на платформе фиксирует эффект Доплера из-за перехода волны из движущейся внутренней среды в внешнюю неподвижную (воздух). Это подчёркивает, что описанный эффект полностью согласуется с наблюдаемой практикой — и может быть объяснён "без прибегания к понятию искривления времени или пространства".

СТО: “Скорость света одинакова для всех наблюдателей, всегда и везде.”
Реграйдерная модель: “Скорость света одинакова — но в каждой своей среде. А среды могут двигаться. И мы это учитываем.”

СТО: “Наблюдатель и источник света — ключевые элементы описания.”
Реграйдерная модель: “Ключевым элементом является среда распространения — независимо от того, движется ли наблюдатель или источник.”

СТО: “Нет нужды в эфире или среде.”
Реграйдерная модель: “Среда не просто нужна — она определяет, где окажется фронт света через секунду.”

Одним из ключевых аспектов предлагаемой модели является то, что она включает в себя специальную теорию относительности как частный случай. В рамках СТО предполагается, что скорость света одинакова для всех инерциальных наблюдателей, что эквивалентно случаю, когда скорость движения реграйдера Sk = 0 относительно наблюдателя. Таким образом, СТО рассматривает ситуацию, в которой реграйдер неподвижен в системе отсчёта наблюдателя, и, следовательно, не вносит вклад в результирующее смещение фронта световой волны.

Предлагаемая модель не противоречит фундаментальному положению СТО о постоянстве скорости света. Вместо этого она уточняет его: свет распространяется со скоростью C относительно своей локальной среды, а не абстрактного геометрического пространства. В результате — при движении этой среды относительно наблюдателя — может наблюдаться увеличение или уменьшение общего расстояния, проходимого фронтом волны за единицу времени.

Такой подход не нарушает ни один из классических экспериментов:

- Эксперимент Майкельсона–Морли: отсутствие эффекта объясняется тем, что реграйдер (среда) движется вместе с установкой. Нет относительного движения — нет интерференционных смещений.
- Эффект Доплера: возникает не в момент излучения внутри реграйдера, а при выходе света за его пределы. Переход между реграйдерами — это и есть момент, где наблюдается смещение частоты.
- Парадокс с источником света: свет не “уводится” скоростью источника, но его фронт может “увестись” движением реграйдера, как это было бы с волной в движущейся жидкости.

Таким образом, модель сохраняет все предсказания СТО, но уточняет и расширяет физическую интерпретацию, добавляя недостающий элемент — движущуюся среду распространения.

\section*{Экспериментальное предложение: “Задержка света в движущейся установке внутри неподвижного вакуума”}
\addcontentsline{toc}{section}{Экспериментальное предложение: “Задержка света в движущейся установке внутри неподвижного вакуума”}

\subsection*{Цель:}
Проверить, влияет ли движение платформы с излучателем и приёмником внутри большого объёма неподвижного вакуума на общее время прохождения света между ними.

\subsection*{Описание установки:}
\begin{itemize}
    \item Внутри вакуумной камеры (например, в \textbf{Plum Brook Station NASA}) создаётся максимально разреженный вакуум.
    \item Внутрь помещается \textbf{подвижная платформа}, на которой жёстко закреплены:
    \begin{itemize}
        \item Источник света (импульсный лазер),
        \item Приёмник света (ультрабыстрый фемтосекундный датчик).
    \end{itemize}
    \item Платформу разгоняют (например, с помощью пневматического ускорителя или сброса с высоты) до высокой скорости, не нарушая целостность вакуума.
\end{itemize}

\subsection*{Суть гипотезы:}
\begin{itemize}
    \item \textbf{СТО:} свет от источника до приёмника (жёстко закреплённых на платформе) пройдёт одинаковый путь за одинаковое время — вне зависимости от движения платформы.
    \item \textbf{Реграйдерная модель:}
    \begin{itemize}
        \item Свет распространяется в локальном вакууме (реграйдере), неподвижном относительно стен камеры.
        \item Платформа с приёмником движется относительно этой среды.
        \item Свету приходится “догонять” приёмник, что создаёт \textbf{наблюдаемую задержку}.
    \end{itemize}
\end{itemize}

\subsection*{Ключевое отличие:}
\begin{itemize}
    \item Ранее: установки в проводимых экспериментах и вакуум двигались совместно (герметично).
    \item Здесь: движется только установка, а вакуум остаётся неподвижным.
\end{itemize}

\subsection*{Ожидаемый результат по реграйдерной модели:}
\begin{itemize}
    \item При увеличении скорости платформы \textbf{время прохождения света} до приёмника \textbf{увеличится}.
    \item При фиксированной дистанции между источником и приёмником (на платформе) наблюдается \textbf{фемтосекундная задержка}, пропорциональная скорости.
\end{itemize}

\subsection*{Преимущества реализации:}
\begin{itemize}
    \item Простота интерпретации результатов.
    \item Возможность точного контроля геометрии и скорости.
    \item Уникальная проверка гипотезы о влиянии движения сквозь вакуум на поведение света.
\end{itemize}

\subsection*{Заключение:}
\begin{quote}
\textbf{Отличие от эксперимента Майкельсона–Морли:} вакуум не движется с установкой, а остаётся внешней средой.\
\textbf{При положительном результате:} эксперимент может подтвердить реграйдерную модель и показать, что специальная теория относительности — частный случай, когда источник, приёмник и среда находятся в согласованном движении.
\end{quote}

\section*{Экспериментальное предложение: “Реграйдер и световой сдвиг во встречном движении”.}
\addcontentsline{toc}{section}{Экспериментальное предложение: “Реграйдер и световой сдвиг во встречном движении”.}

\subsection*{Цель:}
Проверить, оказывает ли движение локальной вакуумной среды (реграйдера) влияние на общее время прохождения света через неё.

\subsection*{Схема:}
Источник света испускает импульс в направлении приёмника.

Между ними движется герметичный контейнер с вакуумом (“реграйдер”) со скоростью S навстречу импульсу.

Свет входит в переднее окно капсулы, проходит через неё (внутри — идеальный вакуум), и выходит через заднее окно.

За капсулой стоит приёмник (например, фемтосекундный датчик).


\subsection*{Два режима:}
\par
A) Контейнер покоится.
— Луч проходит через него без относительного движения среды.
\par
B) Контейнер движется навстречу свету.
— Свет проходит ту же длину внутри движущегося реграйдера.

\subsection*{Гипотеза по реграйдерной модели:}
Внутри движущегося контейнера свет замедляется относительно внешнего наблюдателя, потому что движется против среды.
➜ Это создаёт дополнительную задержку
➜ Время прихода света к приёмнику в варианте (B) будет больше, чем в (A).

\subsection*{Возможный результат:}
Если фиксируется разница во времени прохождения (пусть даже в пико- или фемтосекундах) — это прямое подтверждение, что свет “чувствует” движение среды (реграйдера).

Если разницы нет — это подтверждение подхода СТО.

\subsection*{Теоретическое отличие от СТО:}
СТО: скорость света одинакова в любом инерциальном движении, контейнер не влияет.
Реграйдер: свет движется в локальной среде — её движение влияет на наблюдаемое поведение волны.

\subsection*{Заключение:}
Этот эксперимент — один из способов проверить, влияет ли движение вакуумной среды на распространение света.
Он может стать ключевым верификационным тестом для реграйдерной модели.

\section*{Практические предсказания реграйдерной модели.}
\addcontentsline{toc}{section}{Практические предсказания реграйдерной модели.}

Предлагаемая модель не только согласуется с экспериментами, но и даёт ряд потенциально проверяемых предсказаний, которые могут быть реализованы в лабораторных или астрофизических условиях:

\subsection*{1. Эффект смещения фронта света в движущейся среде.}
При контролируемом движении среды (например, фотонных кристаллов или волноводов), свет, проходящий в одном и том же временном интервале, должен оказываться на различном расстоянии в зависимости от направления и скорости движения среды. Это может быть измерено с высокой точностью с помощью интерферометрии.

\subsection*{2. Локальное отсутствие доплеровского эффекта в движущейся системе.}
Если источник и приёмник находятся и покоятся внутри одного движущегося реграйдера, то наблюдатель не фиксирует эффект Доплера. Это отличает модель от геометрических интерпретаций СТО и может быть проверено в аналогичных экспериментах со звуком или светом внутри изолированных сред.

\subsection*{3. Эффект смещения частоты при выходе света из движущейся среды.}
При резком переходе из одного реграйдера в другой (например, из движущейся жидкости в воздух), фиксируется изменение частоты волны. Это может быть смоделировано как на оптическом, так и на акустическом уровне, подтверждая физический механизм доплера как границу между средами, а не следствие относительности.

\subsection*{4. Уточнённое моделирование астрофизических объектов.}
Радио- и оптические наблюдения звёзд, галактик и пульсаров могут пересчитываться с учётом движения реграйдера (образованного, например, плазменной оболочкой или зонами межзвёздной среды). Это может объяснять некоторые аномалии в смещении спектров без необходимости пересмотра закона постоянства C.

\subsection*{5. Имитация искривления времени через движение реграйдера.}
В некоторых условиях моделируемое “замедление времени” (например, вблизи массивных объектов) может быть пересчитано через движение реграйдерной среды. Это даёт возможность построения лабораторных аналогов ОТО без гравитационных эффектов — лишь на основе движения среды, как в экспериментах с “замедленным светом”.

Таким образом, модель не просто теоретически обоснована, но и предлагает путь к практическим проверкам и новым экспериментам в области оптики, акустики и астрофизики.

\section*{Философские и методологические выводы.}
\addcontentsline{toc}{section}{Философские и методологические выводы.}
\section*{Гипотезы и предсказания, связанные с тёмной материей и инерцией.}
\addcontentsline{toc}{section}{Гипотезы и предсказания, связанные с тёмной материей и инерцией.}


Реграйдерная модель открывает новый подход к объяснению ряда явлений, традиционно интерпретируемых через призму гравитационного искривления или тёмной материи. В частности:

- Реграйдер как физическая структура: если реграйдер обладает собственной скоростью, он может влиять на поведение волны при прохождении через его границу. Это позволяет физически описать и измерять свойства реграйдера, делая его не абстрактным, а экспериментально доступным объектом.

- Объяснение гравитационных линз без искривления пространства: многие эффекты искривления световых лучей около массивных объектов могут быть интерпретированы как прохождение света через реграйдер с изменяющейся скоростью. Смещение, фазовая задержка и искривление траектории становятся следствием физического взаимодействия с реграйдерной средой.

- Эффект инерции как приобретение реграйдером скорости: если реграйдер, как и объект внутри него, участвует в ускорении (например, при разгоне корабля), это может объяснить, почему наблюдатель внутри реграйдера не фиксирует изменений в поведении света. Возникает гипотеза, что ощущения инерции при разгоне могут быть связаны с “втягиванием” среды (реграйдера) в ускоренное движение. Это требует отдельного рассмотрения и может лечь в основу следующей работы.

Таким образом, реграйдер становится не только средой для распространения волн, но и носителем физического взаимодействия, влияющим на наблюдаемые процессы. Это позволяет построить мост между микрофизикой, астрофизикой и теорией относительности без необходимости прибегать к понятию тёмной материи как «невидимой массы».

Реграйдерная модель восстанавливает физическую интуицию в описании волновых процессов, возвращая ключевую роль среде как носителю распространения. Это позволяет избежать ряда абстрактных конструкций, введённых в рамках релятивистской геометрии, и предлагает более наглядную, экспериментально воспроизводимую основу.

1. Волна без среды — это исключение, а не правило.
Любая волна предполагает среду, в которой она распространяется. Модель реграйдера предлагает физически обоснованную среду для света — не противоречащую наблюдаемым данным и не требующую пересмотра фундаментальных законов.

2. Пространство и время — не обязательные носители.
СТО и ОТО оперируют геометрией пространства-времени как абсолютом. Реграйдерная модель переносит акцент с пространства и времени на свойства среды — её движение, структуру, переходы. Это позволяет отказаться от понятий искривления и замедления как обязательных объяснений.

3. Универсальность подхода.
Принципы, заложенные в реграйдерной модели, могут быть применимы не только к свету, но и к другим типам волн: акустическим, квантовым, гравитационным. Модель становится методологическим фундаментом волнового описания в современной физике.

4. Новый инструментарий для физики.
Модель позволяет строить симуляции, анализировать переходы между системами, пересчитывать параметры на основе движения среды. Она предоставляет возможность предугадывать и детектировать движение реграйдеров, а также выявлять границы переходов между ними. Например, если два луча света проходят схожие траектории, но один из них проходит через движущийся реграйдер, наблюдаемое смещение фронта, изменение фазы или временная задержка позволяют физически "пощупать" присутствие реграйдера.

Это может стать основой для нового метода детектирования реграйдерных структур, в том числе в астрофизике. Предположительно, многие явления, которые сегодня интерпретируются как “искривления пространства” или эффекты тёмной материи, могут быть объяснены прохождением света через границы реграйдеров с различными скоростями и направлениями движения. Таким образом, реграйдеры становятся не абстрактными сущностями, а физически измеримыми структурами с параметрами, поддающимися изучению и моделированию.
Модель позволяет строить симуляции, анализировать переходы между системами, пересчитывать параметры на основе движения среды. Это открывает двери к новым способам моделирования физических процессов, включая астрофизику, квантовую оптику и теорию информации.

5. Отказ от догмы.
Реграйдерная модель не опровергает СТО или ОТО, но показывает, что они могут быть вписаны в более широкую картину, в которой среда — не враг, а союзник в понимании мира. Это возвращение к корням физики, но с новой оптикой и современным языком.

6. “Возможным направлением развития является применение модели к описанию волновых эффектов в гравитационно насыщенных областях (например, вблизи чёрных дыр). Это требует отдельного рассмотрения, включающего возможные параметры реграйдера (такие как плотность, устойчивость и проницаемость), и может быть вынесено в следующую работу.”

7. Возможность рождения волны на границе реграйдеров
В рамках модели допускается, что при переходе волны между реграйдерами с различной скоростью или структурой может происходить частичное перераспределение энергии, выражающееся в смещении фазы, изменении частоты или даже возникновении новой волны. Это может иметь отношение к явлениям, наблюдаемым в полупроводниках и оптоэлектронике, в частности — к генерации фотонов в светодиодах, где переход электронов между различными решётками сопровождается высвобождением квантов света. В рамках реграйдерной модели, это может быть интерпретировано как возбуждение волны на границе двух реграйдерных сред.

Таким образом, представленный подход даёт не только физическую, но и философскую основу для переосмысления механизмов распространения света и волновых процессов в целом.


\section*{Эффект Доплера.}
\addcontentsline{toc}{section}{Эффект Доплера.}

Реграйдерная модель объясняет эффект Доплера так же просто, как и поведение звуковой волны в воздухе:

- Когда источник света движется навстречу внешнему наблюдателю (например, на неподвижной платформе), наблюдается синее смещение.
- Когда источник удаляется — наблюдается красное смещение.

Однако здесь важно уточнение: сам эффект Доплера возникает не в момент испускания света внутри реграйдера, а при переходе волны через границу среды. Именно в этот момент граница между реграйдерами становится новым условным источником для внешнего наблюдателя, и именно здесь фиксируется изменение частоты.

Для наблюдателя внутри движущегося реграйдера (например, внутри корабля), эффект Доплера отсутствует, поскольку источник света и наблюдатель находятся в одной и той же среде. Скорость света внутри реграйдера для него остаётся равной С, без смещения частоты.

Для внешнего наблюдателя, по отношению к которому движется реграйдер, эффект Доплера наблюдается, несмотря на то, что свет по-прежнему распространяется со скоростью С. Смещение частоты объясняется переходом света из движущейся среды (реграйдера) в неподвижную среду наблюдателя.

Аналогичный эффект можно наблюдать со звуком, покидающим движущийся поезд. Независимо от того, происходит ли выход звука через мембрану или напрямую — наблюдатель на платформе фиксирует эффект Доплера из-за перехода волны из движущейся внутренней среды в внешнюю неподвижную (воздух). Это подчёркивает, что описанный эффект полностью согласуется с наблюдаемой практикой — и может быть объяснён "без прибегания к понятию искривления времени или пространства".

\section*{Вакуум как реграйдерная среда.}
\addcontentsline{toc}{section}{Вакуум как реграйдерная среда.}

Постановка концепции:

В реграйдерной модели вакуум перестаёт быть абстрактной пустотой. Он трактуется как реграйдер в его чистом виде — область, в которой отсутствуют дополнительные физические компоненты (такие как воздух или жидкость), но которая всё равно остаётся волновой средой, способной переносить свет.

Вакуум — это не отсутствие среды. Это и есть реграйдер. Только в своей очищенной, максимально прозрачной форме.

Реграйдер не универсален, не неподвижен и не вездесущ. Он ограничен, может перемещаться, иметь скорость и быть обрамлённым объектами. Это делает возможным перенос реграйдера вместе с системой — как оболочку, которая “движется вместе” с тем, что её ограничивает.

Иллюстративные примеры

\subsubsection*{Космический корабль.}
Внутри корабля, даже если создан вакуум, пространство между стенками не является «абсолютной пустотой». Это локальный реграйдер, который движется с той же скоростью, что и сам корабль. Свет внутри такого вакуума распространяется внутри подвижной среды, а не в некоем “глобальном покое”.

\subsubsection*{Колба в большом вакуумном зале.}
Если в большой неподвижной вакуумной комнате движется колба с вакуумом внутри:
свет, проходящий через эту колбу, сначала входит в движущийся реграйдер (внутри колбы),
затем выходит обратно в неподвижный реграйдер комнаты.
На входе и выходе произойдут сдвиги волнового фронта — аналог эффекта Доплера. И не потому, что “движется источник” или “наблюдатель”, а потому, что происходит переход волны между двумя реграйдерами с разной скоростью.

\section*{Главный принцип.}
\addcontentsline{toc}{section}{Главный принцип.}

Вся Вселенная — это набор реграйдеров, перекрывающихся, движущихся, обрамляемых.

Это касается не только вещества, но и “пустоты”.

Пространство между молекулами — это реграйдер.

Космос между звёздами — это реграйдер.

Вакуум в приборе — это реграйдер.

И все они могут двигаться, переноситься, иметь границы и "густоту".

\section*{Следствия и предсказания.}
\addcontentsline{toc}{section}{Следствия и предсказания.}

При переходе света между разными областями вакуума с разной “скоростью обрамления” (например, из движущейся колбы в неподвижную камеру), будут наблюдаться фазовые смещения и эффекты аналогичные доплеровским.

Это позволяет переосмыслить многие оптические и астрофизические явления без обращения к “искривлению пространства” — достаточно учесть границы и движение реграйдеров.

Возможно, даже микроволновой фон и его вариации — это следствие реликтового реграйдера вакуума, образовавшегося при раннем расширении Вселенной.

\section*{Вакуум - вид реграйдер.}
\addcontentsline{toc}{section}{Вакуум - вид реграйдер.}

Вакуум — это не ничто. Это реграйдер, освобождённый от дополнительного содержания. Он физически реален, имеет поведение, взаимодействует со светом, и может быть локализован и ограничен и иметь скорость.

В реграйдерной модели вакуум становится фундаментальным участником физики, а не её фоновым “молчаливым” наблюдателем. Он движется, он участвует, он влияет.

\section*{Связь с преобразованиями Лоренца.}
\addcontentsline{toc}{section}{Связь с преобразованиями Лоренца.}

Преобразования Лоренца, лежащие в основе специальной теории относительности, сохраняют математическую форму и внутри реграйдерной модели, однако получают иное физическое наполнение. В классической СТО они трактуются как результат геометрии пространства-времени при предположении о постоянстве скорости света во всех инерциальных системах.

В реграйдерной модели скорость света также остаётся постоянной, но в пределах своей локальной среды. Таким образом, преобразования Лоренца можно интерпретировать как частный случай реграйдерной модели при условии:

- движение реграйдера отсутствует Sk = 0, либо
- наблюдатель и источник света находятся внутри одного и того же реграйдера.

Это открывает возможность построения реграйдерной кинематики — обобщения преобразований Лоренца, в которых учитывается движение среды как физического носителя волны. Такой подход может дополнить или переосмыслить механизмы замедления времени, сокращения длин и других эффектов, предлагая альтернативу геометризации физических процессов.

\section*{Основные определения.}
\addcontentsline{toc}{section}{Основные определения.}

\subsubsection*{Реграйдер.}
Локализованная волновая среда, в которой распространяется свет. В отличие от классического эфира, реграйдер может иметь движение, границы, быть обрамлённым телами и переноситься вместе с ними. Представляет собой активного участника волнового процесса.

\subsubsection*{Локальная скорость света C.}
Постоянная скорость распространения световой волны в пределах одного реграйдера. Независима от движения источника или наблюдателя внутри данной среды.

\subsubsection*{Скорость реграйдера Sk.}
Скорость движения реграйдера (как среды) относительно внешнего наблюдателя. Учитывается в расчётах полного пути, проходимого светом за заданное время.

\subsubsection*{Граница реграйдера.}
Переходная зона между двумя реграйдерами, обладающими различными скоростями, направлениями или структурой. Именно на границе возникает эффект Доплера, а также возможны фазовые и частотные смещения.

\subsubsection*{Реграйдерная система.}
Система объектов, находящихся внутри одного и того же реграйдера. Для таких наблюдателей свет всегда имеет скорость C, а эффекты относительности отсутствуют.

\subsubsection*{Реграйдерное смещение (или фазовый сдвиг).}
Изменение фазы или частоты волны при её переходе между реграйдерами. Может проявляться в виде эффекта Доплера или временной задержки.

\subsubsection*{Реграйдерная интерференция.}
Наблюдаемое смещение волнового фронта при прохождении света через несколько реграйдеров с различными характеристиками. Может имитировать эффекты гравитационного линзирования, искривления пространства и др.

\subsubsection*{Реграйдерная прозрачность (опционально, для будущих работ).}
Гипотетическая характеристика среды, описывающая её способность передавать или поглощать волну. Может использоваться для описания поведения света в экстремальных условиях (например, около чёрных дыр).

\subsubsection*{Вакуум (в контексте модели).}
Реграйдер, лишённый физических компонентов (таких как газы или твёрдые тела), но сохраняющий свойства волновой среды. Вакуум не является “ничем” — это активная, переносимая реграйдерная область.

\subsubsection*{Скорость вакуума (гипотетически).}
Скорость движения вакуума трактуется как скорость реграйдера, в котором он локализован в данный момент. Поскольку вакуум не является абсолютной пустотой, а представляет собой предельно “очищенную” реграйдерную среду, он может обладать ненулевой скоростью, зависящей от движения систем, которые его обрамляют. Например, вакуум внутри космического корабля движется вместе с кораблём.

\section*{Свод ключевых формул и понятий.}
\addcontentsline{toc}{section}{Свод ключевых формул и понятий.}

Основные параметры:
\par
C
\par
скорость света в локальной среде (реграйдере)
\par
Sk
\par
скорость движения реграйдера относительно внешнего наблюдателя
\par
\ell
\par
расстояние, проходимое светом внутри реграйдера
\par
\theta
\par 
угол между направлением распространения света и вектором скорости реграйдера
\par
Время прохождения участка в реграйдере:
\[
t = \ell / C
\]

Расстояние, пройденное светом относительно внешнего наблюдателя за это время:
\[
L_1 = (C + S_k) \cdot (\ell / C)
\]

Оставшееся время и путь в неподвижной среде:
\[
t_2 = 1 - (\ell / C)
\]
\[
L_2 = C \cdot (1 - \ell / C)
\]

Суммарное расстояние за единицу времени:
\[
S_t_o_t_a_l = L_1 + L_2 = C + (\ell \cdot S_k) / C
\]

Обобщённая векторная формула:
\[
S_t_o_t_a_l = C + (\ell \cdot \left|S_k\right| \cdot \cos(\theta)) / C
\]

Эти выражения описывают ключевой принцип:
- свет распространяется с постоянной скоростью C в своей локальной среде, но суммарное расстояние, проходимое за фиксированное время, зависит от скорости и направления движения этой среды относительно наблюдателя.

\subsubsection*{Обобщённая форма для произвольного времени наблюдения.}
Ранее рассматривался случай, когда свет двигался в течение одной секунды. Однако для общей применимости модели — в том числе при переходе к преобразованиям координат и длительным процессам — необходимо обобщение на произвольное наблюдаемое время T_t_o_t_a_l.

Пусть:
\par
T_t_o_t_a_l 
\par
общее время распространения света (наблюдаемое),
\par
\ell
\par 
путь в реграйдере,
\par
Sk 
\par
скорость реграйдера,
\par
C 
\par
скорость света в среде,
\par
\theta
\par 
угол между направлением движения света и вектором скорости реграйдера.
\par
Тогда:

Время прохождения света внутри реграйдера:
\[
t_1 = \ell / C
\]

Оставшееся время вне реграйдера:
\[
t_2 = T_{\text{total}} - (\ell / C)
\]

Путь внутри реграйдера (наблюдаемый снаружи):
\[
L_1 = (C + S_k) \cdot t_1 = (C + S_k) \cdot (\ell / C)
\]

Путь вне реграйдера:
\[
L_2 = C \cdot t_2 = C \cdot (T_{\text{total}} - \ell / C)
\]

Суммарное расстояние, пройденное светом:
\[
S_t_o_t_a_l = L_1 + L_2 = (C + S_k) \cdot (\ell / C) + C \cdot (T_{\text{total}} - \ell / C) = C \cdot T_{\text{total}} + (\ell \cdot S_k) / C
\]

Обобщённая векторная формула с направлением:
\[
S_t_o_t_a_l = C \cdot T_{\text{total}} + (\ell \cdot \left|S_k\right| \cdot \cos(\theta)) / C
\]

Таким образом, реграйдерная модель сохраняет форму и при переходе к произвольным временам наблюдения. Это делает её пригодной для описания систем с длительным движением, в том числе при анализе преобразований координат между различными реграйдерами.

Это — фундамент векторной волновой реграйдерной модели света.
\end{document}