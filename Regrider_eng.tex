\documentclass[12pt]{article}
\usepackage[a4paper,margin=2.5cm]{geometry}
\usepackage{amsmath,amssymb}
\usepackage{graphicx}
\usepackage[utf8]{inputenc}
\usepackage[english]{babel}
\usepackage{hyperref}
\title{The Regrider Model of Light: A Wave Interpretation of the Local Propagation Medium}
\author{Serhii Hliievyi \\ Ukraine \\ \href{mailto:glievoy.s@gmail.com}{glievoy.s@gmail.com}}
\date{}
\begin{document}
\maketitle

\begin{abstract}
    A vector wave model of light propagation in a local moving medium — the regrider — is proposed. Light is considered as a wave maintaining a constant speed $C$ relative to its local medium, which itself may be in motion. The presented model includes the special theory of relativity as a particular case, arising when $S_k = 0$, where $S_k$ is the velocity of the local medium (regrider) relative to an external observer. Physical analogies are described, the mathematical model is formalized, and experimental predictions are provided — including scenarios involving stationary vacuum. The model may be useful for revisiting the wave picture of light and expanding the theoretical toolkit in optics, astrophysics, and fundamental physics.
    \end{abstract}    

\section*{Introduction}
\addcontentsline{toc}{section}{Introduction}
    
In classical and relativistic physics, the propagation of light is traditionally described as a process that does not require a medium. According to the special theory of relativity (STR), the speed of light in vacuum is constant and does not depend on the motion of the source or the observer. However, this model excludes the concept of a medium as a physical carrier of the wave, placing it in logical conflict with the very notion of a wave.
    
This work proposes an alternative, yet experimentally compatible approach: treating light as a wave phenomenon that propagates within a local moving medium — the “regrider”.
    
\textit{Regrider} (from English “regrider”, combining “re-” — again, and “grid” — structure, mesh) is a local, motion-responsive system in which light propagates. Unlike the historical concept of the “aether”, the regrider is not treated as a universal, immobile substance, but rather as a transferable, localized wave medium associated with a specific system (e.g., a spacecraft, train, etc.).
    
Within the proposed model, the fundamental principle of the constancy of the speed of light \textit{relative to its local medium (regrider)} is preserved. However, the total distance that light travels per unit of time may vary depending on the motion of the medium itself.
    
\section*{Physical Analogy}
\addcontentsline{toc}{section}{Physical Analogy}

To illustrate the logic of the proposed model, an analogy is used involving the propagation of sound in moving media. Specifically, the following situations are considered:

\par
1. A man stands on a moving train platform holding a balloon, and the train begins to move: the balloon does not follow him immediately — similarly, light does not directly follow a moving source.
\par
2. Sound emitted inside the moving train propagates through the air, which is stationary relative to the train. Consequently, an observer outside the train may detect a difference in the speed of sound propagation (relative to the external medium).
\par
3. A membrane built into the train wall vibrates when struck by a sound wave and generates a wave in the external environment (air). For the external observer, the membrane becomes the new source of sound — as the boundary between two media. This highlights a key point: the boundary between media acts as a new emitter.
\par
4. If the membrane is removed, and the sound travels directly from the inner medium to the outer one, a similar effect is observed — as when light transitions from one regrider to another, such as from a spacecraft into vacuum.
\par
These examples gradually lead us to the conclusion that when a wave transitions between different media, it is not the wave itself that changes, but the conditions of its propagation — and this becomes the determining factor for the external observer.


\section*{The Regrider Medium Model}
\addcontentsline{toc}{section}{The Regrider Medium Model}

Light is considered as a wave propagating through the regrider, which can have motion relative to an external observer. Inside the regrider, light moves at a constant speed $C$, but for an external observer — who registers not only the light itself, but also the boundary of the medium — the total path of light may appear longer (or shorter), depending on the velocity vector of the regrider.

\section*{Mathematical Model of Light Propagation in the Regrider}
\addcontentsline{toc}{section}{Mathematical Model of Light Propagation in the Regrider}

Let light travel a segment of length $\ell$ within a regrider moving at velocity $S_k$ relative to an external observer. Inside the regrider, light always propagates at speed $C$, and therefore the time $t$ required to traverse this segment is:

\[
t = \ell / C
\]

During this time, the external observer perceives the wavefront as shifting not only due to the motion of light itself, but also due to the motion of the medium. Thus, the distance traveled by light over time $t$ as perceived by the observer is:

\[
L_1 = (C + S_k) \cdot (\ell / C)
\]

After exiting the regrider, the light continues to propagate in a new medium, assumed to be stationary, during the remaining time:

\[
t_2 = 1 - (\ell / C)
\]

And the distance covered during this time will be:

\[
L_2 = C \cdot t_2 = C \cdot (1 - \ell / C)
\]

The total distance covered by light in one unit of time relative to the external observer:

\[
S_{\text{total}} = L_1 + L_2 = C + (\ell \cdot S_k) / C
\]

This is the key expression of the model: light may travel more or less than $C$ in one second, even though its local wave speed remains constant.
\par
If we introduce the direction of the medium's motion via the angle $\theta$ between the direction of the regrider's velocity and the propagation vector of light, we obtain a generalized form:

\[
S_{\text{total}} = C + (\ell \cdot \left|S_k\right| \cdot \cos(\theta)) / C
\]

The values $\cos(\theta) = +1$ and $-1$ represent special cases: the regrider moving strictly in the direction of light propagation and strictly in the opposite direction, respectively.
\par
This forms the basis of the vector model of light propagation in the regrider medium.

\section*{Comparison with the Relativistic Model and Experiments}
\addcontentsline{toc}{section}{Mathematical Model of Light Propagation in the Regrider}
\section*{The Doppler Effect in the Context of the Regrider Model}
\addcontentsline{toc}{section}{The Doppler Effect in the Context of the Regrider Model}

The regrider model explains the Doppler effect as naturally as the behavior of a sound wave in air:

- When a light source moves toward an external observer (e.g., standing on a stationary platform), a "blue shift" is observed.
- When the source recedes, a "red shift" occurs.

However, it is important to clarify that the Doppler effect does not arise at the moment of light emission inside the regrider, but rather when the wave passes through the boundary of the medium. At this point, the boundary between regriders becomes a new effective emitter for the external observer, and this is where the frequency shift is registered.

For an observer located within a moving regrider (e.g., inside a spacecraft), the Doppler effect is absent, since both the light source and the observer reside in the same medium. The speed of light within the regrider remains equal to $C$ for them, with no frequency shift.

For an external observer, relative to whom the regrider is moving, the Doppler effect does appear — even though the light still propagates at speed $C$. The frequency shift is explained by the transition of light from a moving medium (the regrider) into the stationary medium of the observer.

A similar effect can be observed with sound leaving a moving train. Regardless of whether the sound exits through a membrane or directly, an observer on the platform registers a Doppler shift due to the transition of the wave from a moving internal medium into the stationary external one (air). This confirms that the described effect fully corresponds to observed phenomena — and can be explained without invoking the notion of “curved time” or “spacetime distortion.”

\textbf{STR:} “The speed of light is the same for all observers, always and everywhere.”\\
\textbf{Regrider model:} “The speed of light is constant — but within its own local medium. And the media can move. We take that into account.”

\textbf{STR:} “The observer and the source are the key elements of description.”\\
\textbf{Regrider model:} “The key element is the propagation medium — regardless of whether the observer or source moves.”

\textbf{STR:} “There is no need for ether or a medium.”\\
\textbf{Regrider model:} “The medium is not just needed — it determines where the light front will be one second later.”

One of the key aspects of the proposed model is that it includes special relativity as a partial case. In STR, it is assumed that the speed of light is the same for all inertial observers, which corresponds to the case where the regrider’s velocity $S_k = 0$ relative to the observer. In this way, STR considers a situation in which the regrider is stationary in the observer's reference frame and thus does not affect the final displacement of the wavefront.

The proposed model does not contradict the fundamental postulate of STR about the constancy of the speed of light. Rather, it refines it: light propagates at speed $C$ relative to its local medium, not relative to abstract geometric space. As a result, when this medium moves relative to the observer, the total distance traversed by the wavefront in one unit of time may increase or decrease.

This approach does not violate any classical experiments:

- \textit{Michelson–Morley experiment:} The absence of an effect is explained by the fact that the regrider (medium) moves together with the setup. No relative motion — no interferometric shift.
- \textit{Doppler effect:} Arises not at the moment of emission inside the regrider, but when the light exits it. The transition between regriders is the moment where frequency shift occurs.
- \textit{Source paradox:} The light is not “dragged” by the velocity of the source, but its front may shift due to the movement of the regrider — as would be the case for a wave in a moving fluid.

Thus, the model preserves all the predictions of STR, but clarifies and expands its physical interpretation by adding the missing element — the moving propagation medium.

\section*{Experimental Proposal: “Light Delay in a Moving Setup Inside a Stationary Vacuum”}
\addcontentsline{toc}{section}{Experimental Proposal: “Light Delay in a Moving Setup Inside a Stationary Vacuum”}

\subsection*{Objective:}
To test whether the movement of a platform with a light emitter and receiver inside a large stationary vacuum volume affects the total time of light propagation between them.

\subsection*{Setup Description:}
\begin{itemize}
    \item Inside a vacuum chamber (e.g., at \textbf{Plum Brook Station NASA}), a highly rarefied vacuum is created.
    \item A \textbf{mobile platform} is placed inside, with the following elements rigidly attached:
    \begin{itemize}
        \item A light source (impulse laser),
        \item A light receiver (ultrafast femtosecond detector).
    \end{itemize}
    \item The platform is accelerated (e.g., using a pneumatic launcher or by free fall) to high speed, without disturbing the integrity of the vacuum.
\end{itemize}

\subsection*{Hypothesis:}
\begin{itemize}
    \item \textbf{STR:} light from the source to the receiver (rigidly attached to the platform) will travel the same path in the same time, regardless of platform motion.
    \item \textbf{Regrider model:}
    \begin{itemize}
        \item Light propagates through a local vacuum (regrider), which is stationary relative to the chamber walls.
        \item The platform with the receiver moves relative to this medium.
        \item The light must “catch up” to the moving receiver, resulting in an \textbf{observable delay}.
    \end{itemize}
\end{itemize}

\subsection*{Key Distinction:}
\begin{itemize}
    \item Previously: in conducted experiments, both the installation and vacuum moved together (sealed systems).
    \item Here: only the platform moves, while the vacuum remains stationary.
\end{itemize}

\subsection*{Expected Result According to the Regrider Model:}
\begin{itemize}
    \item As the platform speed increases, the \textbf{light travel time} to the receiver will \textbf{increase}.
    \item With a fixed distance between source and receiver (on the platform), a \textbf{femtosecond-scale delay} should be observed, proportional to the speed.
\end{itemize}

\subsection*{Implementation Advantages:}
\begin{itemize}
    \item Simplicity of interpreting results.
    \item Ability to precisely control geometry and speed.
    \item A unique test of the hypothesis that motion through vacuum affects light behavior.
\end{itemize}

\subsection*{Conclusion:}
\begin{quote}
\textbf{Distinct from the Michelson–Morley experiment:} here, the vacuum does not move with the setup but remains an external medium.\\
\textbf{If confirmed:} this experiment could validate the regrider model and demonstrate that special relativity is a partial case — one where the source, receiver, and medium move in unison.
\end{quote}

\section*{Experimental Proposal: “Regrider and Light Shift in Head-On Motion”}
\addcontentsline{toc}{section}{Experimental Proposal: “Regrider and Light Shift in Head-On Motion”}

\subsection*{Objective:}
To test whether the motion of a local vacuum medium (regrider) affects the total time of light propagation through it.

\subsection*{Setup:}
A light source emits an impulse toward a receiver.

Between them, a sealed container filled with vacuum (a “regrider”) moves at speed $S$ in the opposite direction of the light impulse.

The light enters through the front window of the capsule, passes through it (inside — an ideal vacuum), and exits through the rear window.

A receiver (e.g., a femtosecond detector) is located behind the capsule.

\subsection*{Two Modes:}
\par
A) The container is stationary.  
— The beam passes through it without relative motion of the medium.
\par
B) The container moves toward the light.  
— The beam travels the same length inside a moving regrider.

\subsection*{Hypothesis According to the Regrider Model:}
Inside a moving container, light slows down relative to the external observer because it propagates against the motion of the medium.  
➜ This creates an additional delay.  
➜ The time of arrival at the receiver in case (B) will be greater than in case (A).

\subsection*{Possible Result:}
If a difference in travel time is registered (even on the pico- or femtosecond scale), this would be direct confirmation that light “feels” the motion of the medium (regrider).

If no difference is observed — it supports the framework of STR.

\subsection*{Theoretical Distinction from STR:}
\textbf{STR:} The speed of light is the same in any inertial motion; the container has no effect.  
\textbf{Regrider model:} Light propagates in a local medium — and the medium's motion influences the observed wave behavior.

\subsection*{Conclusion:}
This experiment is one of the ways to test whether the motion of a vacuum medium affects the propagation of light.  
It could become a key verification test for the regrider model.

\section*{Practical Predictions of the Regrider Model}
\addcontentsline{toc}{section}{Practical Predictions of the Regrider Model}

The proposed model not only aligns with existing experiments but also provides a number of potentially verifiable predictions that can be tested in laboratory or astrophysical conditions:

\subsection*{1. Front shift effect in a moving medium.}
With controlled motion of the medium (e.g., photonic crystals or waveguides), light propagating over the same time interval should end up at different positions depending on the direction and velocity of the medium's movement. This can be measured with high precision using interferometry.

\subsection*{2. Local absence of the Doppler effect in a moving system.}
If the source and receiver are both stationary within a single moving regrider, the observer will not register a Doppler effect. This distinguishes the model from geometric interpretations of STR and can be tested in analogous sound or light experiments within isolated media.

\subsection*{3. Frequency shift effect when light exits a moving medium.}
When transitioning sharply from one regrider to another (e.g., from a moving liquid to air), a frequency shift of the wave is registered. This can be modeled on both optical and acoustic levels, confirming the Doppler effect as a boundary phenomenon between media, rather than a result of relativity.

\subsection*{4. Refined modeling of astrophysical objects.}
Radio and optical observations of stars, galaxies, and pulsars can be recalculated taking into account the movement of the regrider (formed, for example, by plasma envelopes or interstellar regions). This may explain some spectral shift anomalies without requiring revision of the law of constant $C$.

\subsection*{5. Simulation of time dilation via regrider motion.}
In certain conditions, the modeled "time slowing" (e.g., near massive objects) may be recalculated via the motion of the regrider medium. This opens up the possibility for constructing laboratory analogs of general relativity effects without gravity — based solely on medium motion, as in “slow light” experiments.

Thus, the model is not only theoretically grounded, but also offers a pathway to practical testing and new experimental approaches in optics, acoustics, and astrophysics.

\section*{Philosophical and Methodological Conclusions}
\addcontentsline{toc}{section}{Philosophical and Methodological Conclusions}
\section*{Hypotheses and Predictions Related to Dark Matter and Inertia}
\addcontentsline{toc}{section}{Hypotheses and Predictions Related to Dark Matter and Inertia}

The regrider model opens a new approach to explaining a number of phenomena traditionally interpreted through the lens of gravitational curvature or dark matter. In particular:

- \textbf{The regrider as a physical structure:} if the regrider possesses its own velocity, it can influence wave behavior when crossing its boundary. This allows one to physically describe and measure the properties of the regrider, making it not an abstract entity, but an experimentally accessible object.

- \textbf{Gravitational lensing without spacetime curvature:} many effects of light ray bending near massive objects can be interpreted as light passing through a regrider with changing velocity. The observed deflection, phase delay, and trajectory bending arise as physical interactions with the regrider medium.

- \textbf{Inertial effect as the acquisition of velocity by the regrider:} if the regrider — like the object within it — undergoes acceleration (e.g., during a spacecraft’s thrust), this may explain why an observer inside does not detect changes in light behavior. A hypothesis emerges that sensations of inertia during acceleration may be related to the “pulling” of the medium (regrider) into accelerated motion. This requires separate investigation and may become the basis of a future study.

Thus, the regrider becomes not only a medium for wave propagation, but also a carrier of physical interaction, influencing observable processes. It bridges microphysics, astrophysics, and relativity theory without resorting to the concept of dark matter as "invisible mass".

The regrider model restores physical intuition in the description of wave phenomena, returning a key role to the medium as the carrier of propagation. This allows us to bypass a number of abstract constructions introduced in relativistic geometry and offers a more visual and experimentally reproducible foundation.

\textbf{1. A wave without a medium is the exception, not the rule.}\\
Every wave presupposes a medium through which it propagates. The regrider model offers a physically grounded medium for light — one that does not contradict observations and does not require a revision of fundamental laws.

\textbf{2. Space and time are not obligatory carriers.}\\
STR and GTR operate on the geometry of spacetime as an absolute. The regrider model shifts the emphasis from space and time to properties of the medium — its motion, structure, and transitions. This removes the necessity of curvature and time dilation as mandatory explanations.

\textbf{3. Universality of the approach.}\\
The principles underlying the regrider model can be applied not only to light, but to other types of waves: acoustic, quantum, gravitational. The model becomes a methodological foundation for wave-based description in modern physics.

\textbf{4. A new toolkit for physics.}\\
The model enables simulation, analysis of transitions between systems, and recalculation of parameters based on medium motion. It provides tools to predict and detect regrider motion, as well as to identify boundaries between them. For example, if two light beams follow similar paths, but one passes through a moving regrider, the observed front shift, phase change, or time delay allows the physical “detection” of the regrider’s presence.

This may form the basis for a new method of detecting regrider structures, including in astrophysics. It is conceivable that many phenomena currently interpreted as “spacetime curvature” or effects of dark matter can instead be explained as light passing through regrider boundaries with varying velocities and directions of motion. Thus, regriders become not abstract entities, but physically measurable structures with parameters that can be studied and modeled.

The model also allows for simulations and parameter adjustments across systems, opening doors to new ways of modeling physical processes, including astrophysics, quantum optics, and information theory.

\textbf{5. Rejection of dogma.}\\
The regrider model does not refute STR or GTR, but shows that they can be embedded in a broader picture, where the medium is not an enemy, but an ally in understanding the universe. This is a return to the roots of physics — but with a new lens and modern language.

\textbf{6. A possible direction of development is the application of the model to describe wave effects in gravitationally saturated regions (e.g., near black holes). This requires separate analysis, including possible parameters of the regrider (such as density, stability, and permeability), and may be presented in a future work.}

\textbf{7. Possibility of wave generation at the boundary of regriders.}\\
Within the model, it is allowed that when a wave transitions between regriders with differing velocities or structures, partial redistribution of energy may occur — expressed as phase shift, frequency change, or even the generation of a new wave. This may relate to phenomena observed in semiconductors and optoelectronics, particularly photon generation in LEDs, where electron transitions between different lattices result in photon emission. Within the regrider model, this may be interpreted as excitation of a wave at the boundary of two regrider media.

Thus, the proposed approach provides not only a physical, but also a philosophical basis for rethinking the mechanisms of light propagation and wave phenomena in general.


\section*{Doppler Effect}
\addcontentsline{toc}{section}{Doppler Effect}

The regrider model explains the Doppler effect as intuitively as the behavior of a sound wave in air:

- When a light source moves toward an external observer (e.g., standing on a stationary platform), a blue shift is observed.
- When the source moves away — a red shift is observed.

However, it is important to note that the Doppler effect does not arise at the moment of light emission inside the regrider, but at the moment the wave crosses the boundary of the medium. At that point, the boundary between regriders becomes a new effective emitter for the external observer, and this is where the frequency change is registered.

For an observer inside a moving regrider (e.g., inside a spaceship), the Doppler effect is absent, since the light source and the observer are within the same medium. The speed of light inside the regrider remains equal to $C$, with no frequency shift.

For an external observer, relative to whom the regrider is moving, the Doppler effect is present — even though light still propagates at speed $C$. The frequency shift is explained by the transition of light from the moving medium (regrider) into the stationary medium of the observer.

A similar effect can be observed with sound exiting a moving train. Regardless of whether the sound exits through a membrane or directly, an observer on the platform detects a Doppler effect due to the wave transitioning from the moving internal medium into the stationary external one (air). This underscores that the described effect aligns fully with observed reality — and can be explained without invoking the notion of "time dilation" or "spacetime curvature".

\section*{Vacuum as a Regrider Medium}
\addcontentsline{toc}{section}{Vacuum as a Regrider Medium}

\textbf{Conceptual Foundation:}

In the regrider model, vacuum ceases to be an abstract void. It is interpreted as a regrider in its purest form — a region lacking additional physical components (such as air or liquid), yet still remaining a wave-carrying medium capable of transmitting light.

Vacuum is not the absence of medium. It \textit{is} the medium — the regrider, only in its cleanest, most transparent form.

A regrider is not universal, not immobile, and not omnipresent. It is bounded, can move, has velocity, and can be enclosed by physical structures. This makes it possible for the regrider to be transported along with a system — like a shell that “moves with” what constrains it.

\subsubsection*{Spacecraft}
Inside a spacecraft, even if a vacuum is created, the space between the walls is not an “absolute void.” It is a local regrider moving at the same velocity as the spacecraft itself. Light in such a vacuum propagates within a moving medium — not in some “global rest frame.”

\subsubsection*{A Flask in a Large Vacuum Chamber}
If a flask containing vacuum moves within a large stationary vacuum room:  
light passing through the flask first enters the moving regrider (inside the flask),  
and then exits into the stationary regrider of the room.  
At the entry and exit points, shifts in the wavefront occur — analogous to the Doppler effect.  
And not because the “source” or “observer” moves, but because the wave crosses between two regriders with different velocities.

\section*{Core Principle}
\addcontentsline{toc}{section}{Core Principle}

The entire universe is a set of regriders — overlapping, moving, and bounded.

This applies not only to matter, but also to “emptiness”:

The space between molecules — is a regrider.  
The cosmos between stars — is a regrider.  
The vacuum in an instrument — is a regrider.

And all of them can move, be transported, have boundaries, and exhibit "density".


\section*{Implications and Predictions}
\addcontentsline{toc}{section}{Implications and Predictions}

When light transitions between different vacuum regions with differing “enclosure velocities” (e.g., from a moving flask into a stationary chamber), phase shifts and Doppler-like effects will be observed.

This allows us to reinterpret many optical and astrophysical phenomena without invoking “spacetime curvature” — it is sufficient to account for the boundaries and motion of regriders.

It is even possible that the cosmic microwave background and its variations are consequences of a relic vacuum regrider formed during the early expansion of the Universe.

\section*{Vacuum as a Type of Regrider}
\addcontentsline{toc}{section}{Vacuum as a Type of Regrider}

Vacuum is not nothing. It is a regrider, freed from additional content. It is physically real, exhibits behavior, interacts with light, and can be localized, bounded, and possess velocity.

In the regrider model, vacuum becomes a fundamental participant in physics — not its passive “silent” backdrop. It moves, it participates, it influences.

\section*{Relation to Lorentz Transformations}
\addcontentsline{toc}{section}{Relation to Lorentz Transformations}

The Lorentz transformations, which form the foundation of the special theory of relativity, retain their mathematical form within the regrider model — but acquire a different physical interpretation. In classical STR, they are treated as geometric consequences of spacetime under the assumption of constant light speed in all inertial frames.

In the regrider model, the speed of light also remains constant — but only within its local medium. Thus, the Lorentz transformations can be interpreted as a partial case of the regrider model under the following conditions:

- The regrider is stationary ($S_k = 0$), or
- The observer and the light source are within the same regrider.

This opens the door to developing a regrider-based kinematics — a generalization of Lorentz transformations that incorporates the motion of the medium as the physical carrier of the wave. Such an approach may complement or even reinterpret the mechanisms of time dilation, length contraction, and other effects, offering an alternative to the geometrization of physical processes.

\section*{Key Definitions}
\addcontentsline{toc}{section}{Key Definitions}

\subsubsection*{Regrider.}
A localized wave medium through which light propagates. Unlike the classical aether, a regrider may have motion, boundaries, be enclosed by physical objects, and move with them. It represents an active participant in the wave process.

\subsubsection*{Local Speed of Light $C$.}
The constant speed at which a light wave propagates within a single regrider. It is independent of the motion of the source or observer within that medium.

\subsubsection*{Regrider Velocity $S_k$.}
The velocity of the regrider (as a medium) relative to an external observer. It is taken into account when calculating the total distance traveled by light over a given time.

\subsubsection*{Regrider Boundary.}
The transitional zone between two regriders that differ in velocity, direction, or structure. It is at this boundary that the Doppler effect arises, along with potential phase or frequency shifts.

\subsubsection*{Regrider System.}
A system of objects all located within the same regrider. For such observers, the speed of light is always $C$, and relativistic effects are absent.

\subsubsection*{Regrider Shift (or Phase Shift).}
A change in the phase or frequency of a wave when it transitions between regriders. It may appear as a Doppler effect or a time delay.

\subsubsection*{Regrider Interference.}
An observable wavefront displacement when light passes through multiple regriders with different characteristics. It can imitate effects such as gravitational lensing, spatial curvature, etc.

\subsubsection*{Regrider Transparency (optional, for future work).}
A hypothetical characteristic of the medium describing its ability to transmit or absorb a wave. It may be used to describe the behavior of light in extreme conditions (e.g., near black holes).

\subsubsection*{Vacuum (in the context of the model).}
A regrider devoid of physical components (such as gases or solids), but still retaining the properties of a wave medium. Vacuum is not “nothing” — it is an active, transferable regrider domain.

\subsubsection*{Vacuum Velocity (hypothetical).}
The velocity of a vacuum is understood as the velocity of the regrider in which it is currently localized. Since vacuum is not an absolute void but represents an “ultra-purified” form of the regrider medium, it may possess a nonzero velocity that depends on the motion of the systems enclosing it. For example, vacuum inside a spacecraft moves along with the ship.

\section*{Summary of Key Formulas and Concepts}
\addcontentsline{toc}{section}{Summary of Key Formulas and Concepts}

\textbf{Main parameters:}
\par
$C$ — speed of light in the local medium (regrider)
\par
$S_k$ — velocity of the regrider relative to the external observer
\par
$\ell$ — distance traveled by light within the regrider
\par
$\theta$ — angle between the direction of light propagation and the velocity vector of the regrider

\textbf{Time of travel through the regrider:}
\[
t = \ell / C
\]

\textbf{Distance traveled by light relative to the external observer during this time:}
\[
L_1 = (C + S_k) \cdot (\ell / C)
\]

\textbf{Remaining time and distance in the stationary medium:}
\[
t_2 = 1 - (\ell / C)
\]
\[
L_2 = C \cdot (1 - \ell / C)
\]

\textbf{Total distance per unit time:}
\[
S_{\text{total}} = L_1 + L_2 = C + (\ell \cdot S_k) / C
\]

\textbf{Generalized vector formula:}
\[
S_{\text{total}} = C + (\ell \cdot \left|S_k\right| \cdot \cos(\theta)) / C
\]

These expressions describe the core principle:
\begin{quote}
Light propagates at a constant speed $C$ within its local medium, but the total distance traveled over a fixed period depends on the speed and direction of the medium’s motion relative to the observer.
\end{quote}

\subsubsection*{Generalized Form for Arbitrary Observation Time}
Previously, we considered the case where light propagated over one second. However, for general applicability of the model — including the transition to coordinate transformations and extended processes — it is necessary to generalize for arbitrary observed time $T_{\text{total}}$.

Let:
\par
$T_{\text{total}}$ — total observed propagation time of light,
\par
$\ell$ — path traveled in the regrider,
\par
$S_k$ — velocity of the regrider,
\par
$C$ — speed of light in the medium,
\par
$\theta$ — angle between the direction of light propagation and the velocity vector of the regrider.

Then:

\textbf{Time of light propagation within the regrider:}
\[
t_1 = \ell / C
\]

\textbf{Remaining time outside the regrider:}
\[
t_2 = T_{\text{total}} - (\ell / C)
\]

\textbf{Distance inside the regrider (as seen externally):}
\[
L_1 = (C + S_k) \cdot t_1 = (C + S_k) \cdot (\ell / C)
\]

\textbf{Distance outside the regrider:}
\[
L_2 = C \cdot t_2 = C \cdot (T_{\text{total}} - \ell / C)
\]

\textbf{Total distance traveled by light:}
\[
S_{\text{total}} = L_1 + L_2 = (C + S_k) \cdot (\ell / C) + C \cdot (T_{\text{total}} - \ell / C) = C \cdot T_{\text{total}} + (\ell \cdot S_k) / C
\]

\textbf{Generalized vector formula with direction:}
\[
S_{\text{total}} = C \cdot T_{\text{total}} + (\ell \cdot \left|S_k\right| \cdot \cos(\theta)) / C
\]

Thus, the regrider model maintains its form even when extended to arbitrary observation times. This makes it suitable for describing systems involving prolonged motion — including coordinate transformation analysis across multiple regriders.

This represents the foundation of the vector wave regrider model of light.

\end{document}
